
\chapter{ דוגמאות לפתרונות של בעיות פיזיות  }

\section{הטרנספורמציה של החום}
 
\begin{verbatim}
"תרגיל 1"
"חתיכת הקרח שבו יש מסה"
M = 10 kg;
"הוכנסה לכלי. הטמפרטורה של הקרח"
T = -10  \degreeC ;
"מצא את המסה של מים בכלי לאחר הטרנספורמציה. כמות של חום"
q = 20000 kJ
"חום ספציפי לחימום מים היא שווה ל"
c_v = 4.2 kJ/(kg \degreeC);
"חום ספציפי לחימום קרח שווה ל"
c_i = 2.1 kJ/(kg \degreeC);
"חום ההיתוך של קרח הוא שווה ל"
r = 330 kJ/kg;
"חום האידוי של מים הוא שווה"
\lambda = 2300 kJ/kg;
END
\end{verbatim}

\vspace*{-3mm}

\begin{verbatim}
"פתרון תרגיל 1"
SPACE = R64[x];
"  .x מסה לא ידועה של מים מסומן ב  
כמות החום : כדי לחמם את הקרח ל 0 מעלות"
q_1 = M * c_i * (0 - T);
" כדי להמס את הקרח "
q_2 = M * r;
" כדי לחמם את המים ל 100 מעלות "
q_3 = M * c_v * (100 \degreeC);
" לאידוי של המים "
q_4 = (M - x) \lambda;
"  ערך ידוע x נסמן על ידי "
" על ידי הנחה, נקבל את המשוואה"
mass = \solve(q = q_1 + q_2 + q_3 + q_4); 
\print(mass);
\end{verbatim}

\vspace*{-3mm}

\section{קינמטיקה} 

\begin{verbatim}
"תרגיל 2"
" : יש את הצורה (x משוואה קינמטית של תנועה, של נקודה בקו ישר (על גבי ציר ה  
 $x = c_1 + c_2 * t + c_3 * t^3$ "
"הגדר: (1) קואורדינטות של נקודה, (2) המהירות הרגעית 
  את התאוצה הרגעית  "
END
\end{verbatim}

\vspace*{-3mm}

\begin{verbatim}
"פתרון תרגיל 2"
" $t, c_1, c_2, c_3$: נגדיר את הרווחים עם המשתנים"
SPACE = R64[t, c_1, c_2, c_3];
"משוואת התנועה היא"
x = c_1 + c_2 t + c_3 t^3;
"אנחנו יכולים למצוא את המהירות הרגעית"
v = \D_t(x);
"אנחנו יכולים למצוא את התאוצה הרגעית"
a = \D_t(v);
\print(x, v, a);
\end{verbatim}

\vspace*{-3mm}

\begin{verbatim}
"תרגיל 2 ב"
"לפתור את הבעיה הקודמת ברגע של זמן"
t_0 = 2 "שניות"
"עם הערכים המספרים הבאים"
"$c_1=4; c_2=2; c_3=-0.5$."
END
\end{verbatim}

\vspace*{-3mm}

\begin{verbatim}
"פתרון תרגיל 2 ב"
arg = [t_0, 4, 2, -0.5];
x_0 = \value (x, arg);  
v_0 = \value (v, arg);
a_0 = \value (a, arg);
\print(x_0, v_0, a_0);
\end{verbatim}

\vspace*{-3mm}

\section{פיסיקה מולקולרית} 

\begin{verbatim}
"תרגיל 3"
" H באמצע צינור אופקי, הונחה טיפה של כספית באורך" 
" L האוויר נשאב החוצה מהצינור והקצוות של הצינור נאטמו. אורך צינור שווה ל" 
" $L_d$ כאשר הצינור הונח במאונך, טיפה של כספית זזה למטה במרחק של"
"g  תאוצת כוח המשיכה שווה ל"
"$\rho$  הצפיפות הכספית שווה ל"
"?מה הלחץ הראשוני בצינור "
END
\end{verbatim}

\vspace*{-3mm}

\begin{verbatim}
"פתרון תרגיל 3"
" לא ידוע $p_0$ נניח והלחץ הראשוני בצינור"
SPACE = R64[p_0];
"הלחץ בחלקו התחתון של הצינור גדל, ככל שהוסיפה "
"הכספית לרדת, ולכן הלחץ חדש שווה ל"
p_1 = p_0 + \rho * g * H;
".היא נקודת ההצטלבות של הצינור S נניח ש"
":אז הנפח הראשוני של אוויר בחלק התחתון של הצינור שווה "
v_0 = (L/2 - H/2) * S;
"לאחר סיבוב הצינור, נפח האוויר בחלק התחתון של הצינור הוא שווה ל"
v_1 = (L/2 - H/2 - L_d) * S;
" מתקבלת המשוואה Boyle–Mariotte על פי חוק"
initialPressure = \solve(p_0 * v_0 = p_1 * v_1);
\print(initialPressure);
\end{verbatim}

\vspace*{-3mm}

\begin{verbatim}
"תרגיל 3 ב"
"לפתור את הבעיה הקודמת עם הערכים המספרים הבאים"
H = 0.20 m;
L = 1 m;
L_d = 0.10 m;
"תאוצת הכובד שווה ל"
g = 9.8 m/s^2;
\rho = 13600 kg/m^3;
END
\end{verbatim}

\vspace*{-3mm}

\begin{verbatim}
"פתרון תרגיל 3 ב"
p_1 = p_0 + \rho g H;
v_0 = (L/2 - H/2) S; 
v_1 = (L/2 - H/2 - L_d) S;
initialPressure = \solve(p_0 v_0 = p_1 v_1);
\print(initialPressure);
\end{verbatim}
