\chapter{Polynomial computations}

\section{Calculation of the value of a polynomial at the point}
 
To calculate the value of a function at the point you must run 
\comm{value}{(f, [var1, var2, \ldots, varn])},
where $ f $~--- is a polynomial in which  the variables are replaced by the  corresponding values of $ var1, var2, \ldots, varn $.

\underline{Example.}

\vspace*{-2mm}
\begin{verbatim}
SPACE=R[x, y]; 
f=x^2+5x(y^3+x);
g=\value(f, [1, 2]); 
\print(g);
\end{verbatim}
%begindelete

\ex{$SPACE=R[x, y];$\\ 
\hspace*{4mm} $f=x^2+5x(y^3+x);$\\ 
\hspace*{4mm} $g=value(f, [1, 2]); $\\ 
\hspace*{4mm} $print(g);$}
{$g = 46. 00.$}
%enddelete

\section{Factorization of polynomials. Bringing polynomials to the standard form.}

Polynomials are automatically brought to the standard form, which assumes, 
that the first written the leading monomials, and then younger. 
Recall that the order of variables in the ring determines seniority of variables. 

To bring to the standard form of a polynomial can also be run  
\comm{expand}{(f)}, where $f$~--- is a polynomial.

For factoring polynomials you must execute the command  
\comm{factor}{(f)}, where $f$~--- it is a polynomial.


\underline{Example.}

\vspace*{-2mm}
\begin{verbatim}
SPACE=Q[x, y]; 
f= (y^3+x)^2(x+1)^3;
 g=\expand(f);
h=\factor(g);
\print(g,h);
\end{verbatim}
%begindelete

\ex{$SPACE=Q[x, y]; $\\
\hspace*{4mm} $f= (y^3+x)^2(x+1)^3;$\\
\hspace*{4mm} $g= expand(f);$\\
\hspace*{4mm} $h= factor(g);$\\
\hspace*{4mm} $ print(g,h);$}{}
{$g = y^6x^3+3y^6x^2+3y^6x+y^6+2y^3x^4+6y^3x^3+6y^3x^2+2y^3x+x^5+3x^4+3x^3+x^2;$\\
\hspace*{4mm} $h=(x+1)^3(y^3+x)^2;$}

%enddelete


\section{Geometric progression. Summation of polynomial with respect to the variables }

For the summation of polynomial with respect to the variables we must run 
\comm{SumOfPol}{(f, [x, y], [x1, x2, y1, y2])}, where
$ f $~--- a polynomial, $ x, y $ ~ --- variables for summation,
$ x1, x2 $~--- range of summation over $ x $,
$ y1, y2 $~--- range of summation over $ y $.


If intervals of summation on all variables coincide then we can write 
\comm{SumOfPol}{(f, [x, y], [x1, x2])}, where
$ x1, x2 $~--- summation interval for $ x $ and $ y $.

 

\underline{Example.}

\vspace*{-2mm}
\begin{verbatim}
SPACE=R[x, y, z];
f=x^2z+xy+y^3xz;
res=\SumOfPol(f, [x, y], [2, 4, -2, 3]);
\print(res);
\end{verbatim}
%begindelete

\ex{$SPACE=R[x, y, z];$\\
\hspace*{4mm} $f=x^2z+xy+y^3xz;$\\
\hspace*{4mm} $res=SumOfPol(f, [x, y], [2, 4, -2, 3]);$\\
\hspace*{4mm} $print(res);$}
{$res = 417. 00z+27. 00.$}
%enddelete

To convert a polynomial using the formula of the sum of a geometric progression, you must run 
\comm{SearchOfProgression}{(f)}.
This command searches for a geometric progression with the largest number of members in this polynomial. 
Then do it again for the remaining members, and so on. All the detected progression be written as $ S_n = b_1 (q ^ n-1) / (q-1) $, where
$ S_n $~--- sum of the first $ n $ members, $ b_1 $~--- the first term of a geometric progression, $ q $~--- the geometric ratio.
 
\underline{Example. }

\vspace*{-2mm}
\begin{verbatim}
SPACE=R[x, y, z];
f=x^3+x^4+x^5+x^6+x^7+x^8+x^9+x^10+x^11+x^12+x^13;
g=x+x^5+x^9+x^13+xyz+7x^2y^2z^2+7x^3y^3z^3+100xy+x+x^2+x^3+x^4;
f1=\SearchOfProgression(f);
g1=\SearchOfProgression(g);
\print(f1, g1);
\end{verbatim}
%begindelete

\ex{$SPACE=R[x, y, z];$\\
\hspace*{4mm} $f=x^3+x^4+x^5+x^6+x^7+x^8+x^9+x^{10}+x^{11}+x^{12}+x^{13};$\\
\hspace*{4mm} $g=x+x^5+x^9+x^13+xyz+7x^2y^2z^2+7x^3y^3z^3+100xy+x+x^2+x^3+x^4;$\\
\hspace*{4mm} $f1=SearchOfProgression(f);$\\
\hspace*{4mm} $g1=SearchOfProgression(g);$\\
\hspace*{4mm} $print(f1, g1);$}
{$f1 = (x^{14}-x^{3})/(x-1); $\\
\hspace*{4mm} $g1 = ((100. 00yx+x^{13}+x^{9}+x)+(z^{4}y^{4}x^{4}-zyx)/(zyx-1)+(x^{6}-x)/(x-1)+6. 00z^{3}y^{3}x^{3}+6. 00z^{2}y^{2}x^{2}). $}
 %enddelete

\section{Groebner basis of polynomial ideal}  

 The Groebner basis of polynomial ideal may be obtained due to the Bruno Buchberger algorithm \comm{groebnerB}{()}.
 You can obtain the same basis using matrix algorithm, which similar to F4 algorithm  \comm{groebnerB}{()}.
We use reverse lexicographical order. The order of the variables defined in the command SPACE,


\underline{Examples. }

\vspace*{-2mm}
\begin{verbatim}
SPACE=Q[x, y, z]; 
b=\groebnerB(x^4y^3+2xy^2+3x+1, x^3y^2+x^2, x^4y+z^2+xy^4+3);  
\print(b);
\end{verbatim}
%begindelete

\ex{$SPACE=Q[x, y, z];$\\ 
\hspace*{4mm} $b=groebnerB(x^4y^3+2x y^2+3x+1,  x^3y^2+x^2,  x^4y+z^2+x y^4+3); $\\ 
\hspace*{4mm} $print(b);$}
{$b = [z^2-x^4+3x^2+(-10)x+9, y+(-9)x^4+(-3)x^3-x^2+(-81)x+27, x^5+9x^2+(-6)x+1];$}

%enddelete
\begin{verbatim}
SPACE=Z[x, y, z];
b=\groebner(x^4y^3+2xy^2+3x+1, x^3y^2+x^2, x^4y+z^2+x y^4+3);  
\print(b);
\end{verbatim}
%begindelete

\ex{$SPACE=Q[x, y, z];$\\ 
\hspace*{4mm} $b=groebner(x^4y^3+2xy^2+3x+1,  x^3y^2+x^2,  x^4y+z^2+x y^4+3);  $\\ 
\hspace*{4mm} $print(b);$
}
{$b = [z^2-x^4+3x^2+(-10)x+9, x^5+9x^2+(-6)x+1, y+(-9)x^4+(-3)x^3-x^2+(-81)x+27]. $}
%enddelete

\section{Calculations in quotient ring of ideal}

\comm{reduceByGB}{(f, [g_1, \ldots, g_N])} function reduces polynomial $p$ with
given set of polynomial $g_1, \ldots, g_N$.

\begin{verbatim}
SPACE = Q[x, y, z];
p = \reduceByGB(5y^2 + 3x^2 + z^2, [y + x, 5z^2 + 5z]);
\end{verbatim}

%begindelete
\ex{$SPACE = Q[x, y, z];$\\
\hspace*{4mm} $p = \backslash reduceByGB(5y^2 + 3x^2 + z^2, [y + x, 5z^2 + 5z]);$\\
\hspace*{4mm} $-z+8x^2;$
}
%enddelete

In case when second argument isn't a reduced Groebner basis
result depends on positions of polynomials in array:
among all potential reductors the first one will be chosen.

\begin{verbatim}
SPACE = Q[x, y];
NotGB1 = [x + y, x^2 + y^2];
imForNotGBset1 = \reduceByGB(x^2 + y^2, NotGB1);
NotGB2 = [x^2 + y^2, x + y];
imForNotGBset2 = \reduceByGB(x^2 + y^2, NotGB2);
GB = \groebner(x+y, x^2+y^2);  
imForGB  = \reduceByGB(x^2 + y^2, GB);
\print(GB, imForNotGBset1, imForNotGBset2, imForGB);
\end{verbatim}

%begindelete
\ex{
$SPACE=Q[x,y];$\\
$NotGB1=[x+y,x2+y2];$\\
$imForNotGBset1=reduceByGB(x2+y2,NotGB1);$\\
$NotGB2=[x2+y2,x+y];$\\
$imForNotGBset2=reduceByGB(x2+y2,NotGB2);$\\
$GB=groebner(x+y,x2+y2);$\\
$imForGB =reduceByGB(x2+y2,GB);$\\
$print(GB,imForNotGBset1,imForNotGBset2,imForGB);$
} {
$GB = [y+x, x^2];$\\
$imForNotGBset1 = 2x^2;$\\
$imForNotGBset2 = 0;$\\
$imForGB = 0;$
}
%enddelete



\section{Solution of systems of nonlinear algebraic equations}

To obtain the solution of the polynomial system

$\left\{\begin{array}{rcl} p_{1} & = & 0,\\ p_{2} & = & 0,\\ & \ldots\\ p_{N} & = & 0,\\ \end{array} \right. $

use the command \comm{solveNAE}{(p_{1}, p_{2}, \ldots, p_{N})}. 

\begin{verbatim}
SPACE = R[x, y];
\solveNAE(x^2 + y^2 - 4, y - x^2);
\end{verbatim}

\begin{verbatim}
SPACE = R[a, b, c];
S = \solveNAE(a + b + c, a b + a c + b c, a b c - 1);
\end{verbatim}
