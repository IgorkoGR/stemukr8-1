\chapter{Численные алгоритмы}


\section{Вычисление определенных и несобственных интегралов}

\subsection{1. Вычисление определенных интегралов.}

Вычисление определенных интегралов выполняется при помощи метода Гаусса.
Для вычисления определенного интеграла необходимо выполнить команду:
\Nint(f, a, b, epsilon, N);
Где: 
(a,b) - промежуток интегрирования,
f - подинтегральная функция,
epsilon - количество точных десятичных знаков после запятой ( необязательный параметр ),
N - количество точек в формуле Гаусса ( необязательный параметр ).
Последние три параметра можно не указывать.
Точность можно указывать явно (при помощи параметра epsilon), или при помощи константы MachineEpsilon в текущем кольце.



\underline{Примеры. }

\vspace*{-2mm}
\begin{verbatim}
SPACE=R[x];
f = \sin(x);
B = \Nint(f, 0, \pi); 
\print(B);
\end{verbatim}
%begindelete

\ex{$SPACE=R[x]; $\\
\hspace*{4mm} $f = sin(x);$\\
\hspace*{4mm} $B = Nint(f, 0, \pi);$\\
\hspace*{4mm} $\print(B);$\\
}{
$B = 2.00$.} 
%enddelete

\vspace*{-2mm}
\begin{verbatim}
SPACE=R[x];
f = \exp(x)*x;
B = \Nint(f, 0, 1, 4); 
\print(B);
\end{verbatim}
%begindelete

\ex{$SPACE=R[x]; $\\
\hspace*{4mm} $f = \exp(x)*x;$\\
\hspace*{4mm} $B = Nint(f, 0, 1, 4);$\\
\hspace*{4mm} $\print(B);$\\
}{
$B = 1.00$.} 
%enddelete


\subsection{2. Вычисление несобственных интегралов первого рода.}
Для вычисления несобственного интеграла на бесконечном интервале необходимо выполнить команду:
\Nint(f, a, b, [...], epsilon, N);
Где: 
(a,b) - промежуток интегрирования, где любая из границ интегрирования может быть либо конечным числом, либо \pm\infty;
f - подинтегральная функция,
[...] - точки экстремума подинтегральной функции в промежутке (a,b) ( необязательный параметр ),
epsilon - количество точных десятичных знаков после запятой ( необязательный параметр ),
N - количество точек в формуле Гаусса ( необязательный параметр ).
Последние три параметра можно не указывать.


В случае, если точки экстремума не указываются, то корректность результата обеспечивается в том случае,
когда подинтегральная функция монотонная на промежутке интегрирования.


Несобственные интегралы первого рода вычисляются при помощи следующего алгоритма:
Пусть для определенности промежуток интегрирования имеет вид: [a, \infty). 
Считаем интеграл от функции f(x) с шагом 3N. Получаем отрезки: [a, a+3N], [a+3N, a+6N], ...
Когда значение интеграла на очередном отрезке станет меньше значения интеграла на предыдущем отрезке,
шаг увеличивается  в 10 раз.
Вычисление интеграла останавливается, когда значение интеграла на текущем отрезке становится 
меньше значения интеграла на предыдущем отрезке и меньше машинного нуля.

underline{Примеры.  }
\vspace*{-2mm}
\begin{verbatim}
SPACE=R64[x];
f = \exp(-(x-5)^2);
B = \Nint(f, -\infty, \infty); 
\print(B);
\end{verbatim}
%begindelete

\ex{$SPACE=R64[x]; $\\
\hspace*{4mm} $f=\exp(-(x-5)^2);$\\
\hspace*{4mm} $B = Nint(f, -\infty, \infty);$\\
\hspace*{4mm} $print(B);$\\
}{
$B = 1.77;$.} 
%enddelete



\begin{verbatim}
SPACE=R[x];
f = \exp(-x);
B = \Nint(f, 0, \infty); 
\print(B);
\end{verbatim}
%begindelete

\ex{$SPACE=R[x]; $\\
\hspace*{4mm} $f=\exp(-x);$\\
\hspace*{4mm} $B = Nint(f, 0, \infty);$\\
\hspace*{4mm} $print(B);$\\
}{
$B = 1.00;$.} 
%enddelete



\subsection{3. Вычисление несобственных интегралов второго рода.}
Для вычисления интеграла необходимо выполнить команду:
\NInt(f, a, b, [p1, p2, ..., pn], [q1, q2, ..., qm], epsilon, N);
Где:
(a, b) - промежуток интегрирования,
f - подинтегральная функция,
[ p1, p2, ..., pn ] - список точек разрыва функции f принадлежащих промежутку (a, b),
[ q1, q2, ..., qm ] - список точек экстремума функции f принадлежащих промежутку (a, b)
 ( необязательный параметр ),
epsilon - количество точных десятичных знаков после запятой ( необязательный параметр ),
N - количество точек в формуле Гаусса ( необязательный параметр ),
Последние три параметра можно не указывать.

В случае, если точки экстремума не указываются, то корректность результата обеспечивается в том случае,
когда подинтегральное выражение является отношением непрерывной функции и полинома.


\underline{Пример. }

\vspace*{-2mm}
\begin{verbatim}
SPACE=R64[x];
f = 1/(\abs(x)^(0.5));
\Nint(f, -1, 1, [0]); 
\end{verbatim}
%begindelete

\ex{$SPACE=R64[x]; $\\
\hspace*{4mm} $f=1/(\abs(x)^(0.5));$\\
\hspace*{4mm} $Nint(f, -1, 1, [0]);$\\
}{
$B = 4.0;$.} 
%enddelete
