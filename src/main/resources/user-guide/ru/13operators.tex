\chapter{Список операторов}

{\bf Правило образования наименований математических объектов}
\bigskip

Заглавные и строчные буквы всюду различаются. Пользователь может давать любые имена для математических объектов. Однако эти имена не должны совпадать с операторами и константами, которые определены в системе. Кроме того, имена объектов, умножение которых  не коммутативно,  например,  векторов и матриц,  должны начинаться с заглавных латинских букв, а все остальные имена объектов должны начинаться со строчных букв. Это дает возможность сразу после ввода автоматически получать упрощенное выражение.
\bigskip

Приведем список основных операторов системы Mathpar.   

\comm{clean}{}~---  удаление всех введенных имен объектов.  Если в операторе перечислены имена объектов,  то удаляются только объекты с этими именами. 
\bigskip

{\bf Инфиксные арифметические операторы}

{\bf +}~---   сложение;

{\bf -}~---  вычитание;

{\bf /}~---  деление;

{\bf *}~---  умножение (можно использовать пробел вместо знака умножения);

\comm{times}{}~--- некоммутативное умножение. 
\bigskip


\bigskip

{\bf Постфиксные арифметические операторы}


{\bf !}~---  факториал;

{ ${x} \widehat{\ }{\{ \}}$}~---  возведение в степень;
\bigskip

{\bf Инфиксные операторы сравнения.}


$\mathbf{\backslash le}$~---  меньше или равно;

{\bf >}~---  больше;

 $\mathbf{\backslash ge}$~--- больше или равно;

{\bf ==}~---   равно;

$\mathbf{\backslash ne}$~--- неравно. 
\bigskip

{\bf Инфиксные логические операторы}


 $\mathbf{ \backslash lor}$ ~--- дизъюнкция;

 $\mathbf{ \backslash  \&}$~--- конъюнкция;

 $\mathbf{ \backslash neg}$ ~--- отрицание. 


{\bf Основные префиксные операторы}

 \comm{d}{}~--- символ дифференцирования при записи дифференциального уравнения;

 \comm{D}{}~--- производная функции: \comm{D}{(f)} и \comm{D}{(f, x)}~--- первая производная по $x$; $\mathbf D (f, y\widehat{\ }{}3)$~--- третья производная по $y$ и т. д.;

 \comm{expand}{}~--- преобразование выражения в сумму с раскрытием всех скобок в выражении;

 \comm{fullExpand}{}~--- преобразование в сумму выражения, которое содержит логарифмические, показательные и тригонометрические функции;

 \comm{extendedGCD}{}~--- расширенный алгоритм вычисления наибольшего общего делителя (НОД) полиномов.  В результате получается вектор,  содержащий НОД и дополнительные множители аргументов;

 \comm{GCD}{}~--- вычисление НОД полиномов;

 \comm{factor}{}~--- представление выражения в виде произведения;

 \comm{fullFactor}{}~--- представление выражения,  содержащего логарифмические и показательные функции,  в виде произведения;

 \comm{initCond}{}~--- задание начальных условий для системы линейных дифференциальных уравнений;

 \comm{LCM}{}~--- вычисление наименьшего общего кратного (НОК) полиномов;

 \comm{lim}{}~--- предел выражения;

 \comm{print}{}~--- печать выражений. Аргументами выступают имена выражений, разделенные запятыми. Каждое выражение будет печататься на новой строке;

 \comm{printS}{}~--- печать выражений в одну строчку, для перехода на следующую строчку нужно использовать <<$\backslash$n>>;

 \comm{plot}{}~--- построение графика функции, которая задана явно;
 
  \comm{plot3D}{}~--- построение графика функции двух переменных, которая задана явно;

 \comm{paramPlot}{}~--- построение графика функции, которая задана  параметрически; 

\comm{tablePlot}{}~--- построение графика функции,  заданной таблицей аргументов и значений;

 \comm{prod}{}~--- произведение (символ $\prod$);

 \comm{randomPolynom}{}~--- генерация случайного полинома;

 \comm{randomMatrix}{}~--- генерация случайной матрицы;

 \comm{randomNumber}{}~--- генерация случайного числа;

 \comm{sequence}{}~--- задание последовательности;

 \comm{showPlots}{}~--- построение в одной системе координат графиков функций, которые должны быть определены раньше;

 \comm{solveLDE}{}~--- решение систем линейных дифференциальных уравнений;

 \comm{systLAE}{}~--- задание систем линейных алгебраических уравнений;

 \comm{systLDE}{}~--- задание систем линейных дифференциальных уравнений;

 \comm{sum}{}~---  сумма (символ $\sum$);

 \comm{time}{}~--- определение процессорного времени в миллисекундах;

 \comm{value}{}~--- вычисление значение выражения при подстановке заданных выражений или чисел вместо переменных кольца.

 \bigskip
 
 {\bf Операторы процедуры,  ветвления и цикла}

 \comm{procedure}{}~--- оператор объявления процедуры; 
 
\comm{if }{(\ ) \{\  \}} \comm{else }{\{ \ \}}~--- оператор ветвления; 

\comm{while }{( \ ) \{ \ \}}~--- оператор цикла с предусловием; 

\comm{for }{(\ ; \ ; \ ) \{ \ \}}~--- оператор цикла с счетчиком.  
\bigskip

{\bf Матрицы, их элементы и матричные операторы}

[\ , \  ] ~--- задание вектора (строки);

[[\ ,\  ], [\ ,\  ]]~--- задание матрицы;


A\_\{i,j\}~--- (i,j)-элемент матрицы A;  

A\_\{i,?\}~---  строка i матрицы A; 

A\_\{?,j\}~--- столбец j  матрицы A; 

$\backslash$O\_\{n,m\}~--- нулевая матрица размера $n \times m$ ; 

$\backslash$I\_\{n,m\}~---  $n \times m$ матрица  с единицами на главной диагонали; 

+, -, *~--- сложение,  вычитание,  умножение; 

\comm{charPolynom}{}~---  характеристический полином;
 
\comm{kernel}{}~--- ядро оператора (нуль-пространство);

\comm{transpose}{} или  $\mathbf{A \widehat{\ } \{T\}}$~--- транспонированная матрица;
 
\comm{conjugate}{} или  $\mathbf{A\widehat{\ }\{\backslash ast\}}$~--- сопряженная матрица;

\comm{toEchelonForm}{}~---   эшелонная (ступенчатая) форма; 

\comm{det}{}~---   определитель;
 
\comm{inverse}{}  или  $\mathbf{A\widehat{\ }\{-1\}}$~--- обратная матрица;

\comm{adjoint}{}  или  $\mathbf{A\widehat{\ }\{\backslash star\}}$~--- присоединенная матрица;
 
\comm{genInverse}{} или $\mathbf{A\widehat{\ }\{+\}}$~--- обобщенная обратная матрица Мурра-Пенроуза;
 
\comm{closure}{}  или $\mathbf{A\widehat{\ }\{\backslash times\}}$~--- замыкание,  т.е. сумма $I+A+A^2+A^3+\ldots$. Для классических алгебр это эквивалентно $(I-A)^{-1}$; 

\comm{LDU}{} --- LDU-представление матрицы. Результат~--- вектор из трёх матриц $[L,D,U]$. Здесь $L$~--- нижняя треугольная матрица, $U$~--- верхняя треугольная матрица, 
$D$~--- матрица перестановок, умноженная на обратную к диагональной матрицу.

\comm{BruhatDecomposition}{} --- разложение Брюа матрицы. Результат~--- вектор из трёх матриц $[V,D,U]$. Здесь  $V$ и $U$~--- верхние треугольные матрицы, 
$D$~--- матрица перестановок, умноженная на матрицу, которая является обратной к диагональной матрице.
