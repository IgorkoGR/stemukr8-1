\chapter{Вычисления в идемпотентных алгебрах}


\section{Тропические алгебры}
Определены следующие тропические алгебры :\\

ПОЛУПОЛЯ\\
1) На множестве целых чисел ${\mathbb Z}$ определены:\\ 
$ZMaxPlus$,  
$ZMinPlus$.\\
2) На множестве чисел ${\mathbb R}$ определены:\\
$RMaxPlus$, 
$RMinPlus$,  
$RMaxMult$,  
$RMinMult$.\\
3) На множестве чисел ${\mathbb R}64$ определены:\\
$R64MaxPlus$, 
$R64MinPlus$,  
$R64MaxMult$,  
$R64MinMult$.\\

ПОЛУКОЛЬЦА\\
1) На множестве целых чисел ${\mathbb Z}$ определены:\\ 
$ZMaxMin$,
$ZMinMax$,
$ZMaxMult$,
$ZMinMult$.\\
2) На множестве чисел ${\mathbb R}$ определены:\\
$RMaxMin$, 
$RMinMax$.\\
3) На множестве чисел ${\mathbb R}64$ определены:\\
$R64MaxMin$, 
$R64MinMax$.

 

Примеры тропических алгебр: 

SPACE = ZMaxPlus [x,  y,  z]; 

SPACE = R64MinMult [u,  v];  

SPACE = RMaxMin [u,  v]. 

 Пример простой задачи в полукольце $ZMaxMult$.
%begindelete
\smallskip

\underline{Пример 1. }

\vspace*{-3mm}
%enddelete
\begin{verbatim}
SPACE = ZMaxMult[x, y];
a = 2; b = 9; c = a + b; d = a*b; \print(c, d)
\end{verbatim}
%begindelete

Результат выполнения:\\
$c = 9; $\\
$d = 18.$
%enddelete

Помимо сложения и умножения доступна операция замыкания, вызываемая командой $\backslash closure(a)$, где $a$ --- элемент или матрица.
Замыкание $closure(a)=\mathbb{1}\oplus a\oplus a^{2}\oplus\dots$
%begindelete
\smallskip

\underline{Пример 2. }

\vspace*{-3mm}
%enddelete
\begin{verbatim}
SPACE = R64MaxPlus[x, y];
A = [
  [0.0, -0.6, -0.62, -3.76],
[-6.29, 0.0, -0.99, -7.61],
[-2.74, -0.86, 0.0, -3.47],
[-6.31, -5.11, -2.69, 0.0]
];
B = \closure(A);
\print(B); 
\end{verbatim}
%begindelete

Результат выполнения:\\
$$B=
 \left(\begin{array}{cccc}
0.0 & -0.6 & -0.62 & -3.76\\
-3.74 & 0.0 & -0.99 & -4.46\\
-2.74 & -0.86 & 0.0 & -3.47\\
-5.43 & -3.55 & -2.69 & 0.0
\end{array}\right)$$
%enddelete 

В остальных параграфах этой главы приведены примеры задач, которые решаются в тропических алгебрах, являющихся полуполями.
%begindelete
\section{Решение систем линейных алгебраических уравнений}
Команда $\backslash solveLAETropic(A, b)$ позволяет найти частное решение уравнения вида $Ax = b$.
%begindelete
\smallskip

\underline{Пример 3. }

\vspace*{-3mm}
%enddelete
\begin{verbatim}
SPACE = R64MaxPlus[x, y];
A = [
  [1.00, 1.00, 0.00],
  [2.00, 0.00, 3.00],
  [3.00, 4.00, 2.00]
];
b = [8.00, 7.00, 11.00];
X = \solveLAETropic(A, b); 
\print(X);
\end{verbatim}
%begindelete

Результат выполнения:\\
$X = \left(\begin{array}{c}
5.00\\
7.00\\
4.00\\
\end{array}\right)
$ 
%enddelete
\section{Решение систем линейных алгебраических неравенств}
Команда $\backslash solveLAITropic(A,b)$ позволяет найти решение неравенства Ax $\leq$ b.

%begindelete
\smallskip

\underline{Пример 4. }

\vspace*{-3mm}
%enddelete
\begin{verbatim}
SPACE = R64MaxPlus[x, y];
A = [
  [1.00, 1.00, 0.00],
  [2.00, 0.00, 3.00],
  [3.00, 4.00, 2.00]
];
b = [10.00, 7.00, 11.00]; 
X = \solveLAITropic(A, b); 
\print(X);
\end{verbatim}
%begindelete

Результат выполнения:\\
$X=[(-\infty,5.00],(-\infty,7.00],(-\infty,4.00]]$ 
%enddelete

%begindelete
\smallskip

\underline{Пример 5. }

\vspace*{-3mm}
%enddelete
\begin{verbatim}
SPACE = ZMinPlus[x, y];
A = [
  [1, 1, 0],
  [2, 0, 3],
  [3, 4, 2]
];
b = [10, 7, 11];
X = \solveLAITropic(A, b); 
\print(X);
\end{verbatim}
%begindelete

Результат выполнения:\\
$X=[[9,\infty),[9,\infty),[10,\infty)]$ 
%enddelete

\section{Решение  уравнения Беллмана}
\subsection{Однородное уравнение Беллмана}
Команда $\backslash BellmanEquation(A)$ позволяет найти решение однородного уравнения Беллмана $Ax = x$.
%begindelete
\smallskip

\underline{Пример 6.}

\vspace*{-3mm}
%enddelete
\begin{verbatim}
SPACE = R64MaxPlus[x, y];
A = [
  [0.00, -2.00, -\infty, -\infty],
  [-\infty, 0.00, 3.00, -1.00],
  [-1.00, -\infty, 0.00, -4.00],
  [2.00, -\infty, -\infty, 0.00]
]; 
X = \BellmanEquation(A); 
\print(X);
\end{verbatim}
%begindelete

Результат выполнения:\\
$$
X=\left(\begin{array}{cccc}
0.00 & -2.00 & 1.00 & -3.00\\
2.00 & 0.00 & 3.00 & -1.00\\
-1.00 & -3.00 & 0.00 & -4.00\\
2.00 & 0.00 & 3.00 & 0.00
\end{array}\right) \left(\begin{array}{c}
v_{1}\\
v_{2}\\
v_{3}\\
v_{4}
\end{array}\right), \forall v_{1}, v_{2}, v_{3}, v_{4}.$$
%enddelete
\subsection{Неднородное уравнение Беллмана}
Команда $\backslash BellmanEquation(A,b)$ позволяет найти решение неоднородного уравнения Беллмана $Ax\oplus b=x$.
%begindelete
\smallskip

\underline{Пример 7. }

\vspace*{-3mm}
%enddelete
\begin{verbatim}
SPACE = R64MaxPlus[x, y];
A = [
  [0.00, -2.00, -\infty, -\infty],
  [-\infty, 0.00, 3.00, -1.00],
  [-1.00, -\infty, 0.00, -4.00],
  [2.00, -\infty, -\infty, 0.00]
];
b = [[1], [-\infty], [-\infty], [-\infty]]; 
X = \BellmanEquation(A, b); 
\print(X);
\end{verbatim}
%begindelete

Результат выполнения:\\
$$X=
 \left(\begin{array}{cccc}
0.00 & -2.00 & 1.00 & -3.00\\
2.00 & 0.00 & 3.00 & -1.00\\
-1.00 & -3.00 & 0.00 & -4.00\\
2.00 & 0.00 & 3.00 & 0.00
\end{array}\right)
\left(\begin{array}{c}
v_{1}\\
v_{2}\\
v_{3}\\
v_{4}
\end{array}\right)
\oplus\left(\begin{array}{c}
1.00\\
3.00\\
0.00\\
3.00
\end{array}\right), \forall v_{1}, v_{2}, v_{3}, v_{4}.$$
%enddelete
\section{Решение неравенства Беллмана}
\subsection{Однородное неравенство Беллмана}
Команда $\backslash BellmanInequality(A)$ позволяет найти решение однородного неравенства Беллмана $Ax\leq x$.

\subsection{Неднородное неравенство Беллмана}
Команда $\backslash BellmanInequality(A, b)$ позволяет найти решение неоднородного неравенства Беллмана $Ax\oplus b\leq x$.

\section{Нахождение кратчайшего пути между вершинами графа}
\subsection{Вычисление таблицы кратчайших расстояний для всех вершин графа}
Пусть A - матрица расстояний между смежными вершинами ($x_{ii}$=0 $\forall i$; $x_{ij}=\infty$, если нет ребра, соединяющего вершины i и j).
Команда $\backslash searchLeastDistances(A)$ позволяет найти наименьшие расстояния между всеми вершинами графа.
В результате будет получена матрица  кратчайших расстояний между вершинами. 

%begindelete
\smallskip

\underline{Пример 8. }

\vspace*{-3mm}
%enddelete
\begin{verbatim}
SPACE = R64MinPlus[x, y];
A = [
  [0.00, 7.00, 9.00, \infty, \infty, 14.00],
  [7.00, 0.00, 10.00, 15.00, \infty, \infty],
  [9.00, 10.00, 0.00, 11.00, \infty, 2.00],
  [\infty, 15.00, 11.00, 0.00, 6.00, \infty],
  [\infty, \infty, \infty, 6.00, 0.00, 9.00],
  [14.00, \infty, 2.00, \infty, 9.00, 0.00]
];
B = \searchLeastDistances(A);
\print(B);
\end{verbatim}
%begindelete

Результат выполнения:\\
$$B= \left(\begin{array}{cccccc}
0.00 & 7.00 & 9.00 & 20.00 & 20.00 & 11.00\\
7.00 & 0.00 & 10.00 & 15.00 & 21.00 & 12.00\\
9.00 & 10.00 & 0.00 & 11.00 & 11.00 & 2.00\\
20.00 & 15.00 & 11.00 & 0.00 & 6.00 & 13.00\\
20.00 & 21.00 & 11.00 & 6.00 & 0.00 & 9.00\\
11.00 & 12.00 & 2.00 & 13.00 & 9.00 & 0.00
\end{array}\right) $$
%enddelete
\subsection{Нахождение кратчайшего пути между двумя вершинами графа}
Пусть A - матрица расстояний между смежными вершинами ($x_{ii}$=0 $\forall i$; $x_{ij}=\infty$, если нет ребра, соединяющего вершины i и j).
Команда $\backslash findTheShortestPath(A, i, j)$ позволяет найти кратчайший путь между вершинами i и j.
 %begindelete
\smallskip

\underline{Пример 9. }

\vspace*{-3mm}
%enddelete
\begin{verbatim}
SPACE = R64MinPlus[x, y];
A = [
  [0.00, 7.00, 9.00, \infty, \infty, 14.00],
  [7.00, 0.00, 10.00, 15.00, \infty, \infty],
  [9.00, 10.00, 0.00, 11.00, \infty, 2.00],
  [\infty, 15.00, 11.00, 0.00, 6.00, \infty],
  [\infty, \infty, \infty, 6.00, 0.00, 9.00],
  [14.00, \infty, 2.00, \infty, 9.00, 0.00]
];
X = \findTheShortestPath(A, 0, 4);
\print(X);
\end{verbatim}
%begindelete

Результат выполнения:\\
$X=[[0, 2, 5, 4]]$
%enddelete
 
