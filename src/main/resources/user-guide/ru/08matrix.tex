\chapter{Матричные функции}
\section{Вычисление транспонированной матрицы}
Для вычисления транспонированной матрицы для матрицы $A$ необходимо выполнить команду 
 \comm{transpose}{(A)} или $\mathbf{A\widehat{\ } \{T\}}$. 

\underline{Пример. }

\vspace*{-2mm}
\begin{verbatim}
SPACE=Z[x]; 
A=[[1, 2], [4, 5]];
B=A^{T}; 
\print(B);
\end{verbatim}
%begindelete

\ex{$SPACE=Z[x]; $\\
\hspace*{4mm} $A=\left(\begin{array}{cc}1 & 2\\ 4& 5\\ \end{array}\right);$\\
\hspace*{4mm} $B=A^{T}; $\\
\hspace*{4mm} $print(B);$}{
$B =\left(\begin{array}{cc}1 &4 \\ 2 &5 \end{array}\right)$.} 
%enddelete

\section{Вычисление обратной и присоединенной матрицы }
\subsection{Вычисление обратной матрицы}

Для вычисления обратной матрицы для матрицы $A$ необходимо выполнить команду  
 \comm{inverse}{(A)} или $\mathbf{A\widehat{\ } \{(-1)\}}$. 

\underline{Примеры. }

\vspace*{-2mm}
\begin{verbatim}
SPACE=Q[x]; 
A=[[1, 4], [4, 5]];
B=\inverse(A); 
\print(B);
\end{verbatim}
%begindelete

 \ex{$SPACE=Q[x]; $\\
\hspace*{4mm} $A=\left(\begin{array}{cc}1 & 2\\ 4& 5\\ \end{array}\right);$\\
\hspace*{4mm} $B=inverse(A); $\\
\hspace*{4mm} $print(B);$}
{$B =\left(\begin{array}{cc}(-5)/11 &4/11\\ 4/11 & (-1)/11 \end{array}\right);$}

%enddelete
\begin{verbatim}
SPACE=Q[x, y]; 
A=[[x+y, x], [y, \cos(x)]];
B=\inverse(A); 
\print(B);
\end{verbatim}
%begindelete

\ex{$SPACE=Q[x, y];$\\ 
\hspace*{4mm} $A=\left(\begin{array}{cc}x+y& x\\ y& cos(x)\\  \end{array}\right);$\\
\hspace*{4mm} $B=inverse(A); $\\
\hspace*{4mm} $print(B);$
}{
$B =\left(\begin{array}{cc}\frac{\cos(x)}{y\cos(x)+x\cos(x)+(-yx)} &\frac{-x}{y\cos(x)+x\cos(x)+(-yx)}\\ \frac{-y}{(y\cos(x)+x\cos(x)+(-yx)} & y+\frac{x}{(y \cos(x)+x \cos(x)+(-yx)} \end{array}\right).$} 
%enddelete
 
\subsection{Вычисление присоединенной матрицы}
Для вычисления присоединенной матрицы для заданной матрицы $A$ необходимо выполнить команду  
\comm{adjoint}{(A)}
или  $\mathbf{A\widehat{\ }\{\backslash star\}}$. 

\underline{Примеры. }

\vspace*{-2mm}
\begin{verbatim}
SPACE=Z[x]; 
A=[[1, 4], [4, 5]];
B=\adjoint(A);  
\print(B);
\end{verbatim}
%begindelete

\ex{$SPACE=Z[x]; $\\
\hspace*{4mm} $A=\left(\begin{array}{cc}1 & 2\\ 4& 5\\ \end{array}\right);$\\
\hspace*{4mm} $B=adjoint(A); $\\
\hspace*{4mm} $print(B);$}
{$B =\left(\begin{array}{cc}5 & -4 \\ -4 &1\end{array}\right) ;$ }

%enddelete
\begin{verbatim}
SPACE=Z[x, y]; 
A=[[\cos(y), \sin(x)], [\sin(y), \cos(x)]];
B=\adjoint(A); 
\print(B);
\end{verbatim}
%begindelete

 \ex{$SPACE=Z[x, y]; $\\
\hspace*{4mm} $A=\left(\begin{array}{cc} cos(y)& sin(x)\\ sin(y)& cos(x)\end{array}\right);$\\
\hspace*{4mm} $B=adjoint(A);$\\ 
\hspace*{4mm} $print(B);$}{
$B =\left(\begin{array}{cc}cos(x) & -sin(x) \\ -sin(y) &cos(y)\end{array}\right)$.}  
%enddelete

\section{Вычисление определителя матрицы}
Для вычисления определителя матрицы A необходимо выполнить команду   \comm{det}{(A)}. 

\underline{Примеры. }

\vspace*{-2mm}
\begin{verbatim}
SPACE=Z[x]; 
A=[[1, 4], [4, 5]];
B=\det(A); 
\print(B);
\end{verbatim}
%begindelete

\ex{$SPACE=Z[x]; $\\
\hspace*{4mm} $A=\left(\begin{array}{cc}1 & 4\\ 4& 5\\ \end{array}\right);$\\
\hspace*{4mm} $B=det(A); $\\
\hspace*{4mm} $print(B);$}
{$B = -11;$} 

%enddelete

\begin{verbatim}
SPACE=Z[x, y]; 
A=[[x+y, \sin(x)], [y, \cos(x)]];
B=\det(A); 
\print(B);
\end{verbatim}
%begindelete

 \ex{$SPACE=Z[x, y];$\\ 
\hspace*{4mm} $A=\left(\begin{array}{cc}x+y& sin(x)\\ y& cos(x)\end{array}\right);$\\
\hspace*{4mm} $B=det(A);$\\ 
\hspace*{4mm} $print(B);$}{$B = y\cdot cos(x)+x\cdot cos(x)-y\cdot  sin(x)$.}
%enddelete

\section{Вычисление сопряженной матрицы}
Для вычисления сопряженной матрицы необходимо выполнить команду  
 \comm{conjugate}{(A)}  или $\mathbf {A\widehat{\ } \{ \backslash ast\}}$. 

\underline{Пример. }

\vspace*{-2mm}
\begin{verbatim}
SPACE=C[x];
A=[[1+\i, 2-\i], [-3, -2\i]];
B=A^{\ast};
\print(B);
\end{verbatim}
%begindelete

\ex{$SPACE=C[x];$\\
\hspace*{4mm} $A=\left(\begin{array}{cc}1+i& 2-i\\-3& -2i\end{array}\right);$\\
\hspace*{4mm} $B=A^{\ast};$\\
\hspace*{4mm} $print(B);$}
{$
B = \left(\begin{array}{cc}
1-1. 0i& -3  \\
 2+1. 0i& 2. 0i\\ \end{array}\right). $}
%enddelete

\section{Вычисление обобщенной обратной матрицы}

Для вычисления обобщенной обратной матрицы Мурра-Пенроуза необходимо выполнить команду  
\comm{genInverse}{(A)} или $\mathbf{A\widehat{\ } \{+\}}$. 

\underline{Пример. }

\vspace*{-2mm}
\begin{verbatim}
SPACE=Z[x];
A=[[1, 4, 5], [2, 4, 5]];
B=A^{+};
\print(B);
\end{verbatim}
%begindelete

\ex{$SPACE=Z[x];$\\
\hspace*{4mm} $A=\left(\begin{array}{ccc}1& 4& 5\\ 2& 4& 5\end{array}\right);$\\
\hspace*{4mm} $B=A^{+};$\\
\hspace*{4mm} $print(B);$}
{$
B =  \left(\begin{array}{ccc}
-1& 1& 0\\
 8/41&  (-4)/41& 0\\
 10/41& (-5)/41& 0\\ \end{array}\right).$}
%enddelete
 
\section{Вычисление ядра оператора и эшелонной формы}

\subsection{Вычисление эшелонной формы матрицы}
Для вычисления эшелонной формы матрицы A необходимо выполнить команду  
 \comm{toEchelonForm}{(A)}. 

\underline{Примеры. }

\vspace*{-2mm}
\begin{verbatim}
SPACE=Z[x]; 
A=[[1, 4], [4, 5]];
B=\toEchelonForm(A); 
\print(B);
\end{verbatim}
%begindelete

\ex{$SPACE=Z[x]; $\\
\hspace*{4mm} $A=\left(\begin{array}{cc}1 & 4\\ 4& 5\\ \end{array}\right);$\\
\hspace*{4mm} $B=toEchelonForm(A);$\\ 
\hspace*{4mm} $print(B);$}{
$B =\left(\begin{array}{cc}-11 &0 \\ 0 &-11 \end{array}\right);$} 

%enddelete 
\begin{verbatim}
SPACE=Z[x, y]; 
A=[[\cos(y), \sin(x)], [\sin(y), \cos(x)]];
B=\toEchelonForm(A); 
\print(B);
\end{verbatim}
%begindelete 

\ex{$SPACE=Z[x, y]; $\\ 
\hspace*{4mm} $A=\left(\begin{array}{cc} cos(y)& sin(x)\\ sin(y)& cos(x)\end{array}\right);$\\ 
\hspace*{4mm} $B=toEchelonForm(A);$\\ 
\hspace*{4mm} $print(B);$}
{$B =\left(\begin{array}{cc} cos(y) cos(x)- sin(x) sin(y) &0\\ 0 & cos(y) cos(x)- sin(x) sin(y) \end{array}\right)$.} 
%enddelete
 
 \subsection{Вычисление ядра оператора}

 Для вычисления ядра оператора матрицы A необходимо выполнить команду
\comm{kernel}{(A)}. 

\underline{Примеры. }

\vspace*{-2mm}
\begin{verbatim}
SPACE=Z[x]; 
A=[[1, 4], [4, 16]];
B=\kernel(A); 
\print(B);
\end{verbatim}
%begindelete

\ex{$SPACE=Z[x]; $\\
\hspace*{4mm} $A=\left(\begin{array}{cc}1& 4\\ 4& 16 \end{array}\right);$\\
\hspace*{4mm} $B=kernel(A); $\\
\hspace*{4mm} $print(B);$}{
$B =\left(\begin{array}{cc}0 &4\\ 0 &-1 \end{array}\right);$}

 %enddelete
\begin{verbatim}
SPACE=Z[x, y]; 
A=[[x+y, x], [(x+y)x, x^2]];
B=\kernel(A); 
\print(B);
\end{verbatim}
%begindelete

\ex{$SPACE=Z[x, y]; $\\
\hspace*{4mm} $A=\left(\begin{array}{cc}x+y& x\\ (x+y)x& x^2 \end{array}\right);$\\
\hspace*{4mm} $B=kernel(A);$\\ 
\hspace*{4mm} $print(B);$}{
$B =\left(\begin{array}{cc}0 &x\\ 0 &-y-x \end{array}\right)$.} 
%enddelete

\section{Вычисление характеристического полинома матрицы}

Для вычисления характеристического полинома матрицы A, элементы которой из $R[x_1,\ldots,x_m]$, необходимо
задать кольцо полиномов $R[x_1,\ldots,x_m]R[t]$ или $R[t,x_1,\ldots,x_m]$, в котором переменная $t$~--- это
переменная, по которой строится полином, 
и выполнить команду \comm{charPolynom}{(A)}.
Например,если элементы исходной матрицы из кольца Z[x,y],
то  можно указать Z[x,y]Z[t] или Z[t,x,y].

\underline{Примеры. }

\vspace*{-2mm}
\begin{verbatim}
SPACE=Z[x]; 
A=[[1, 4], [4, 5]];
B=\charPolynom(A); 
\print(B);
\end{verbatim}
%begindelete
%Результат выполнения:\\ 
\ex{$SPACE=Z[x]; $\\
\hspace*{4mm} $A=\left(\begin{array}{cc}1 & 2\\ 4& 5\\ \end{array}\right);$\\
\hspace*{4mm} $B=charPolynom(A); $\\
\hspace*{4mm} $print(B);$}
{
$B = x^{2}+(-6)x+(-11);$} 

%enddelete
\begin{verbatim}
SPACE=Z[x, y]Z[t]; 
A=[[\cos(y), \sin(x)], [\sin(y), \cos(x)]];
B=\charPolynom(A); 
\print(B);
\end{verbatim}
%begindelete

\ex{$SPACE=Z[x, y]Z[t]; $\\
\hspace*{4mm} $A=\left(\begin{array}{cc} cos(y)& sin(x)\\ sin(y)& cos(x)\end{array}\right);$\\
\hspace*{4mm} $B=adjoint(A);$\\ 
\hspace*{4mm} $print(B);$}{
$B = t^{2}+(-\cos(x)-\cos(y))t+\cos(y)\cos(x)-\sin(x)\sin(y)$.}  
%enddelete

\section{ LDU-разложение} 
Для вычисления LDU-разложения матрицы $A$, нужно выполнить команду 
 \comm{LDU}{(A)}. 

Результат~--- это три матриц $[L,D,U]$. Здесь $L$~--- нижняя треугольная матрица, $U$~--- верхняя треугольная матрица, 
$D$~--- матрица перестановок, умноженная на матрицу, которая является обратной к диагональной матрице. 
Если элементы матрицы $A$ из коммутативной области $R$, то и элементы матриц $L$, $D^{-1}$, $U$ также принадлежат области $R$.

\underline{Примеры. }

\vspace*{-2mm}
\begin{verbatim}
SPACE=Z[x]; 
A=[[0, 1, 0], [4, 5, 1],[1, 1, 1]];
B=\LDU(A);
\print(B);
\end{verbatim}
%begindelete

\ex{$SPACE=Z[x];$ \\
\hspace*{4mm} $A=\left(\begin{array}{ccc}
0 & 1&0\\
 4& 5&1\\
 1&1&1\\ 
\end{array}\right);$\\
\hspace*{4mm} $B=LDU(A) $\\
\hspace*{4mm} $print(B);$}
{$B = \left[\begin{array}{ccc}
 \left(\begin{array}{ccc}
  4& 0&  0\\
  0& 4& 0\\
  -1& 1&  3\\
\end{array}\right),\ &\ 
\left(\begin{array}{ccc}
  0&1/16 &      0\\
  1/4& 0& 0\\
  0& 0&   1/12\\
\end{array}\right),\ &\ 
\left(\begin{array}{ccc}
  4& 5&  1\\
  0& 4& 0\\
  0& 0& 3\\
\end{array}\right) 
 \end{array}\right].$
} 

%enddelete
\begin{verbatim}
SPACE=Z[x]; 
A=[[1, 4,0,1], [4, 5,5,3],[1,2,2,2],[3,0,0,1]];
B=\LDU(A);
\print(B);
\end{verbatim}
%begindelete

\ex{$SPACE=Z[x];$ \\
\hspace*{4mm} $A=\left(\begin{array}{cccc}1 & -4&0&1\\ 4& 5&5&3\\ 1&2&2&2\\ 3&0&0&1\\ \end{array}\right);$\\
\hspace*{4mm} $B=LDU(A) $\\
\hspace*{4mm} $print(B);$}
{$B = \left[\begin{array}{c}
\left(\begin{array}{cccc}1& 0&   0&   0  \\
 4& -11& 0&   0 \\
 1& -2&  -12& 0  \\
 3& -12& 60&  -60\\ 
\end{array}\right) \\ 
\null \\ 
\left(\begin{array}{cccc}1& 0&         0&       0      \\
 0& 1/(-11)& 0&       0      \\
 0& 0&         1/132& 0      \\
 0& 0&         0&       1/720\\ 
\end{array}\right) \\ 
\null \\
\left(\begin{array}{cccc}1& 4&   0&   1  \\
 0& -11& 5&   -1 \\
 0& 0&   -12& -13\\
 0& 0&   0&   -60\\ 
\end{array}\right)\\
 \end{array}\right].$
} 

%enddelete
\begin{verbatim}
SPACE=Z[x,y];
A=[[\cos(y),\sin(x)],[\sin(y),\cos(x)]];
B=\LDU(A);
\print(B);
\end{verbatim}
%begindelete

\ex{$SPACE=Z[x,y];$ \\
\hspace*{4mm} $A=\mtr{\cos(y)}{\sin(x)}{\sin(y)}{\cos(x)}$\\
\hspace*{4mm} $B=LDU(A) $\\
\hspace*{4mm} $print(B);$}
{$B = \left[\begin{array}{c}
\mtr{\cos(y)}{0}{\sin(y)}{\cos(y)\cos(x)+(-\sin(x)\sin(y))}\\
\null \\
\mtr{1/\cos(y)}{0}{0}{1/((\cos(y))^2\cos(x)+(-1\cos(y)\sin(x)\sin(y)))}\\
\null \\
\mtr{\cos(y)}{\sin(x)}{0}{\cos(y)\cos(x)+(-\sin(x)\sin(y))}\\
 \end{array}\right].$}
%enddelete

\section{Разложение Брюа}
Для вычисления разложения Брюа матрицы $A$, нужно выполнить команду 
 \comm{BruhatDecomposition}{(A)}. 

Результат~--- вектор из трёх матриц $[V,D,U]$. Здесь  $V$ и $U$~--- верхние треугольные матрицы, 
$D$~--- матрица перестановок, умноженная на матрицу, которая является обратной к диагональной матрице. Если элементы матрицы $A$ из коммутативной области $R$, то и элементы матриц $V$, $D^{-1}$, $U$ также принадлежат области $R$.


\underline{Примеры. }

\vspace*{-2mm}
\begin{verbatim}
SPACE=Z[x]; 
A=[[1, 4,0,1], [4, 5,5,3],[1,2,2,2],[3,0,0,1]];
B=\BruhatDecomposition(A);
\print(B);
\end{verbatim}
%begindelete

\ex{$SPACE=Z[x];$ \\
\hspace*{4mm} $A=\left(\begin{array}{cccc}1 & -4&0&1\\ 4& 5&5&3\\ 1&2&2&2\\ 3&0&0&1\\ \end{array}\right);$\\
\hspace*{4mm} $B=BruhatDecomposition(A) $\\
\hspace*{4mm} $print(B);$}
{$B = \left[\begin{array}{c}
\left(\begin{array}{cccc}-24& 0&  12& 1\\
 0&   60& 15& 4\\
 0&   0&  6&  1\\
 0&   0&  0&  3\\ 
\end{array}\right) \\ 
\null \\
\left(\begin{array}{cccc}0&     0&      1/(-144)& 0          \\
 0&     0&      0&          1/(-1440)\\
 0&     1/18& 0&          0          \\
 1/3& 0&      0&          0          \\ 
\end{array}\right) \\ 
\null \\
\left(\begin{array}{cccc}3& 0& 0&   1  \\
 0& 6& 6&   5  \\
 0& 0& -24& -16\\
 0& 0& 0&   60 \\ 
\end{array}\right) \\ 
 \end{array}\right].$
} 

%enddelete
\begin{verbatim}
SPACE=Z[x,y];
A=[[\cos(y),\sin(x)],[\sin(y),\cos(x)]];
B=\BruhatDecomposition(A);
\print(B);
\end{verbatim}
%begindelete

\ex{$SPACE=Z[x,y];$ \\
\hspace*{4mm} $A=\mtr{\cos(y)}{\sin(x)}{\sin(y)}{\cos(x)}$\\
\hspace*{4mm} $B=BruhatDecomposition(A) $\\
\hspace*{4mm} $print(B);$}
{$B = \left[\begin{array}{c}
\mtr{-\cos(y)*\cos(x)+\sin(x)*\sin(y)}{\cos(y)}{0}{\sin(y)}\\
\null \\
\mtr{0}{1/(-\cos(y)\sin(y)\cos(x)+\sin(x)(\sin(y))^2)}{1/\sin(y)}{0}\\
\null \\
\mtr{\sin(y)}{\cos(x)}{0}{-\cos(y)\cos(x)+\sin(x)\sin(y)}\\
 \end{array}\right].$
} 
%enddelete

\section{Решение задач линейного программирования}
Пусть задана целевая функция $\sum_{j = 1}^n c_j x_j$
и условия
$$\sum_{j = 1}^n a_{ij}x_j\leqslant b_i,\text{ где }i = 1,2,\ldots,m,$$
$$x_j\geqslant 0,\text{ где }j = 1,2,\ldots,n.$$

Определим $m\times n$-матрицу $A = (a_{ij})$,
$m$-мерный вектор $b = (b_i)$, $n$-мерный вектор $c = (c_j)$ и $n$-мерный вектор $x = (x_j)$.

Тогда целевую функцию можно записать в виде 
$c^Tx,$
а условия~--- в виде 
$$Ax \leqslant b,$$
$$ x \geqslant 0.$$

Для решения задач линейного программирования нужно выполнить команду 
 \comm{SimplexMax}{} или  \comm{SimplexMin}{}. Результат~--- вектор $x$.

В зависимости от вида задачи возможны следующие варианты вызова команд.

1. Для решения задачи
$$c^Tx \rightarrow max$$
при условиях
$$Ax \leqslant b,$$
$$ x \geqslant 0,$$
используем команду \comm{SimplexMax}{(A, b, c)}.

Если целевую функцию надо минимизировать, т.е. $$c^Tx \rightarrow min,$$ то используем команду \comm{SimplexMin}{(A, b, c)}.

%begindelete
\underline{%enddelete
Пример.%begindelete
}
%enddelete

Максимизировать $$3x_1 + x_2 + 2x_3$$
при условиях
$$%begindelete
  \left\{
  \begin{array}{c}%enddelete
   x_1 +  x_2 + 3x_3 \leqslant 30,\ \\
  2x_1 + 2x_2 + 5x_3 \leqslant 24,\ \\
  4x_1 +  x_2 + 2x_3 \leqslant 36,\ \\
   x_1,   x_2,   x_3 \geqslant 0.%begindelete
  \end{array}
  \right.%enddelete
$$

\begin{verbatim}
  SPACE = R64[];
  A = [[1, 1, 3],[2, 2, 5],[4, 1, 2]];
  b = [30, 24, 36];
  c = [ 3,  1,  2];
  x = \SimplexMax(A, b, c);
\end{verbatim}

%begindelete

\ex{}
{[8.0, 4.0, 0.0];}
%enddelete

2. Для решения задачи
$$c^Tx \rightarrow max$$
при условиях
$$A_1 x \leqslant b_1,$$
$$A_2 x =         b_2,$$
$$ x \geqslant 0,$$
используем команду \comm{SimplexMax}{(A_1,A_2, b_1, b_2, c)}.

Если целевую функцию надо минимизировать, т.е. $$c^Tx \rightarrow min,$$ то используем команду \comm{SimplexMin}{(A_1,A_2, b_1, b_2, c)}.

%begindelete
\underline{%enddelete
Пример.%begindelete
}
%enddelete

 Максимизировать $$7x_1 + x_3 - 4x_4$$

при условиях
$$%begindelete
  \left\{
  \begin{array}{c} %enddelete
   x_1 - x_2 + 2x_3 - x_4 \leqslant  6,\ \\
  2x_1 + x_2 - x_3 = -1,\ \\
   x_1,  x_2,   x_3,  x_4 \geqslant  0. %begindelete
  \end{array}
  \right. %enddelete
$$

\begin{verbatim}
  SPACE = R64[];
  A1 = [[1, -1,  2, -1]];
  A2 = [[2,  1, -1,  0]];
  b1 = [ 6];
  b2 = [-1];
  c  = [7, 0, 1, -4];
  x = \SimplexMax(A1, A2, b1, b2, c);
\end{verbatim}
%begindelete
\ex{}
{[0.8, 0.0, 2.6, 0.0];}
%enddelete

3. Для решения задачи
$$c^Tx \rightarrow max$$
при условиях
$$A_1 x\leqslant b_1,$$
$$A_2 x =         b_2,$$
$$A_3 x\geqslant b_3,$$
используем команду \comm{SimplexMax}{(A_1,A_2, A_3,b_1, b_2, b_3,c)}.

Если целевую функцию надо минимизировать, т.е. $$c^Tx \rightarrow min,$$ то используем команду \comm{SimplexMin}{(A_1,A_2, A_3,b_1, b_2, b_3, c)}.

%begindelete
\underline{%enddelete
Пример.%begindelete
}
%enddelete

Максимизировать $$x_1 + x_2$$

при условиях
$$%begindelete
  \left\{
  \begin{array}{c} %enddelete
   4x_1 -  x_2 \leqslant  8,\ \\
   2x_1 +  x_2 \leqslant 10,\ \\
  -5x_1 + 2x_2 \geqslant -2,\ \\
   x_1,    x_2 \geqslant  0.%begindelete
  \end{array}
  \right. %enddelete
$$

\begin{verbatim}
  SPACE = R64[];
  A1 = [[ 4, -1], [2,  1]];
  A3 = [[-5,  2]];
  b1 = [ 8, 10];
  b3 = [-2];
  c  = [1, 1];
  x = \SimplexMax(A1, [[]], A3, b1, [], b3, c);
\end{verbatim}
%begindelete
\ex{}
{[2.0, 6.0];}
%enddelete

4. Для решения   задачи общего вида, как и в предыдущем пункте,
$$c^Tx \rightarrow max$$ можно обойтись четырьмя параметрами.
Можно задать матрицу $A$, вектор  $b$, целевую функцию $c$  и
использовать команду \comm{SimplexMax}{(A,signs,b,c)}, где
массив целых чисел $signs$ определяет знаки сравнения:
        -1  обозначает меньше или равно , 0 обозначает равно  и 1  обозначает больше или равно.
Должно быть в $signs$ столько же чисел, сколько элементов в векторе $b$.

Если целевую функцию надо минимизировать, т.е. $$c^Tx \rightarrow min,$$ то используем команду \comm{SimplexMin}{(A,signs,b,c)}.

%begindelete
\underline{%enddelete
Пример.%begindelete
}
%enddelete

 Минимизировать $$-2x_1-4x_2-2x_3$$
при условиях
$$%begindelete
  \left\{
  \begin{array}{c}%enddelete
  -2x_1 +  x_2 +  x_3 \leqslant 4,\ \\
  - x_1 +  x_2 + 3x_3 \leqslant 6,\ \\
    x_1 - 3x_2 +  x_3 \leqslant 2,\ \\
    x_1,   x_2,   x_3 \geqslant 0.%begindelete
  \end{array}
  \right.%enddelete
$$
In:
\begin{verbatim}
  SPACE = R64[];
  A = [[-2,  1, 1],[-1,  1, 3],[ 1, -3, 1]];   
  b = [ 4,  6,  2];
  c = [-2, -4, -2]; 
  signs = [-1, -1, -1];
  x = \SimplexMin(A, signs, b, c);
\end{verbatim}
%begindelete
\ex{}
{ Simplex:\ This problem has no finite solution.}
%enddelete



%begindelete
\section{Контрольные задания}
В системе Mathpar для матриц $A\in \mathbb{Q}, \ B\in \mathbb{Z}/(17)\mathbb{Z}[x, y] $ $$A =\left(\begin{array}{ccc}3&0 &7\\ 0 &-5&5\\4&17&2\\ \end{array}\right), \ 
B =\left(\begin{array}{ccc}0 &x&y+1\\ x^2&0 &-y-x \\ x^2+y&-x &-3yx\\ \end{array}\right)$$ вычислите 
\begin{itemize}
 \item транспонированную матрицу, 
 \item определитель, 
 \item обратную матрицу, 
 \item присоединенную матрицу, 
 \item характеристический полином, 
 \item разложение Брюа, 
 \item LDU разложение, 
 \item ядро оператора и эшелонную форму. 
 \end{itemize}
%enddelete
